\begin{abstract}

Structure-based drug discovery is a process for both hit finding and optimization. It relies on a validated three-dimensional model of a target biomolecule, used to rationalise structure-function relationship for this particular target. 
Recently, a concept of an ultra-large virtual screening has emerged for the rapid discovery of high-affinity hit compounds. In this concept, an ultra-large library consisting of already-available or rapidly synthesizable compounds is screened for a target of interest; and then the highest-scored compounds are filtered and clustered by similarity; and finally, few hundreds of them are validated. It has been shown that this approach can yield high-affinity (from sub-micromolar to low-nanomolar) hits for various targets, including G-protein coupled receptors and COVID-19 main protease MPro. However, such screening requires substantial computational resources (hundreds of years of CPU time for a billion-sized library, or tens of thousands of dollars according to cloud platform prices), making it inaccessible for many academic groups worldwide. 
Here we show that it is possible to exploit active learning that relies on simple linear regression models to accelerate such virtual screening and retrieve up to 90\% of the top-1\% of the docking hit-list after docking of only 10\% of the ligands.
Previously, it was proven that such acceleration is possible using models based on deep learning approaches, most of which require access to high-end GPUs and substantial training and inference time. In our work, we show that to predict the results of the intrinsically inaccurate ligand docking with low ligand sampling depth, it is enough to use simple linear regression models, and subsequent improvement in model complexity does not improve the prediction quality. Moreover, we explore the active learning meta-parameters, and find that constant batch size models with simple ensembling method provide the best ligand retrieval rate.
In general, this work provides a simple approach for the computationally accessible accelerated virtual screening.
We hope that it can serve as a blueprint for the future design of low-compute agents for exploration of the chemical space via large-scale accelerated docking. Coupled with recent breakthroughs in protein structure prediction, this can significantly increase accessibility of such methods for the academic community, providing us with the rapid discovery of high-affinity hit compounds for various targets.

\end{abstract}
