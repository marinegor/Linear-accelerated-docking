\section{Discussion}

In this work, we propose a domain-specific active learning model for acceleration of the docking-based ultra-large library screening. Compared to previous works in the field \cite{deepdocking, logistic_regression, Yang2021_shoichet_active_learning, Graff2021AcceleratingLearning}, we focus on different aspects of such approach, such as machine learning model complexity, and influence of information exchange between models trained for different docking batches.

Given exponentially growing chemical databases \cite{warr_exploration_2022, lyu_modeling_2023}, we focus on the performance of the simplest models such as linear regression, that have been optimized to be computationally efficient for both training and inference on extremely large datasets. Robust performance of the linear methods shown here, allows one to use vector databases \cite{noauthor_vector_nodate, noauthor_qdrant_nodate, johnson2019_faiss_vector_database} for the pre-computed fingerprints, and extract and compute molecule 3D structures and subsequently dock them only when needed, drastically decreasing the computational requirements for the actual docking.

As shown by comparison with the second docking for CB2 and AA2AR datasets, ligand docking (especially with a low sampling depth) is relatively inaccurate, and is able to retrieve only half of the ligands from the dataset in the active learning regime after docking of the first 10\%. We believe that this intrinsic inaccuracy of this method is what gives the simple linear models such robustness in this case. Due to the limited sampling depth, the docking scores are not accurate enough, so that any complex model will not be able to predict their true value. Hence, model complexity can be reduced drastically without loosing its performance.

Also, we note the importance of a random train-test split of the molecules, even though for the chemical datasets it is often important to perform a similarity-aware data split. For small-molecule docking, chemically similar molecules often have similar docking scores, and even require clustering after the screening to select hits for experimental validation. Hence, a machine learning model task is to find similar molecules to the already docked ones with high docking score. However, a simple chemical similarity is not enough for this task, as can be seen by poor performance of the k-neighbours models (Figure \ref{fig:fig_3_singleshot}). But a linear regression can learn which fingerprint bits are more important than others, and reliably extract such molecules for the subsequent docking. 

Even though for the linear regression the model inference is fast, the actual active learning regime that incorporates memory might substantially increase the inference time for the later stages of the screening, when the number of models for ensembling is large. However, in this work, we show that the simple MeanRank regime that averages molecule ranks for each model, shows superior performance. Luckily, for this type of ensembling, a base model's results can be cached for later, without the need to re-run the model inference at each stage.

Moreover, here we show that state-of-the-art performance can be achieved even without the need for GPU computational resources, which were employed by all but one previous works either in the message-passing neural networks \cite{Graff2021AcceleratingLearning}, as DeepChem models \cite{Yang2021_shoichet_active_learning}, or as multi-layer perceptron \cite{deepdocking}. This can significantly increase the availability of the method. Compared to the preprint \cite{logistic_regression} that also focuses on reducing the accelerated virtual screening costs, we show that incorporation of simple memory mechanisms into the active learning model can substantially increase its performance on both early and late stages of the screening.
