\section{Conclusion}

In this work, we demonstrate that the active learning, based on linear regression models, is well suited for the accelerated screening of ultra-large virtual libraries via structure-based docking. Using few diverse datasets as an example, we show that linear models, such as linear regression and linear support vector machine, show performance comparable with much more computationally requiring models, such as random forest. Our benchmarks demonstrate that models with a small batch size of $10\ 000$ molecules perform better at the active learning regime, which decreases the potential model training requirements and time. We show substantial decrease in computational time, retrieving, for various datasets, 48-91\% of top-1\% of the ligands after docking 10\% of the library, and 85-98\% after docking 30\% of the library. We hypothesize that such a robust performance of linear models is coupled with intrinsic inaccuracy of low sampling depth small molecule docking. Finally, we envision the appearance of active learning agents based on linear regression, that will greatly democratize access to this approach for the academic community. 